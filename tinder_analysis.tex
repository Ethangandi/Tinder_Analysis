% Options for packages loaded elsewhere
\PassOptionsToPackage{unicode}{hyperref}
\PassOptionsToPackage{hyphens}{url}
%
\documentclass[
]{article}
\usepackage{amsmath,amssymb}
\usepackage{iftex}
\ifPDFTeX
  \usepackage[T1]{fontenc}
  \usepackage[utf8]{inputenc}
  \usepackage{textcomp} % provide euro and other symbols
\else % if luatex or xetex
  \usepackage{unicode-math} % this also loads fontspec
  \defaultfontfeatures{Scale=MatchLowercase}
  \defaultfontfeatures[\rmfamily]{Ligatures=TeX,Scale=1}
\fi
\usepackage{lmodern}
\ifPDFTeX\else
  % xetex/luatex font selection
\fi
% Use upquote if available, for straight quotes in verbatim environments
\IfFileExists{upquote.sty}{\usepackage{upquote}}{}
\IfFileExists{microtype.sty}{% use microtype if available
  \usepackage[]{microtype}
  \UseMicrotypeSet[protrusion]{basicmath} % disable protrusion for tt fonts
}{}
\makeatletter
\@ifundefined{KOMAClassName}{% if non-KOMA class
  \IfFileExists{parskip.sty}{%
    \usepackage{parskip}
  }{% else
    \setlength{\parindent}{0pt}
    \setlength{\parskip}{6pt plus 2pt minus 1pt}}
}{% if KOMA class
  \KOMAoptions{parskip=half}}
\makeatother
\usepackage{xcolor}
\usepackage[margin=1in]{geometry}
\usepackage{color}
\usepackage{fancyvrb}
\newcommand{\VerbBar}{|}
\newcommand{\VERB}{\Verb[commandchars=\\\{\}]}
\DefineVerbatimEnvironment{Highlighting}{Verbatim}{commandchars=\\\{\}}
% Add ',fontsize=\small' for more characters per line
\usepackage{framed}
\definecolor{shadecolor}{RGB}{248,248,248}
\newenvironment{Shaded}{\begin{snugshade}}{\end{snugshade}}
\newcommand{\AlertTok}[1]{\textcolor[rgb]{0.94,0.16,0.16}{#1}}
\newcommand{\AnnotationTok}[1]{\textcolor[rgb]{0.56,0.35,0.01}{\textbf{\textit{#1}}}}
\newcommand{\AttributeTok}[1]{\textcolor[rgb]{0.13,0.29,0.53}{#1}}
\newcommand{\BaseNTok}[1]{\textcolor[rgb]{0.00,0.00,0.81}{#1}}
\newcommand{\BuiltInTok}[1]{#1}
\newcommand{\CharTok}[1]{\textcolor[rgb]{0.31,0.60,0.02}{#1}}
\newcommand{\CommentTok}[1]{\textcolor[rgb]{0.56,0.35,0.01}{\textit{#1}}}
\newcommand{\CommentVarTok}[1]{\textcolor[rgb]{0.56,0.35,0.01}{\textbf{\textit{#1}}}}
\newcommand{\ConstantTok}[1]{\textcolor[rgb]{0.56,0.35,0.01}{#1}}
\newcommand{\ControlFlowTok}[1]{\textcolor[rgb]{0.13,0.29,0.53}{\textbf{#1}}}
\newcommand{\DataTypeTok}[1]{\textcolor[rgb]{0.13,0.29,0.53}{#1}}
\newcommand{\DecValTok}[1]{\textcolor[rgb]{0.00,0.00,0.81}{#1}}
\newcommand{\DocumentationTok}[1]{\textcolor[rgb]{0.56,0.35,0.01}{\textbf{\textit{#1}}}}
\newcommand{\ErrorTok}[1]{\textcolor[rgb]{0.64,0.00,0.00}{\textbf{#1}}}
\newcommand{\ExtensionTok}[1]{#1}
\newcommand{\FloatTok}[1]{\textcolor[rgb]{0.00,0.00,0.81}{#1}}
\newcommand{\FunctionTok}[1]{\textcolor[rgb]{0.13,0.29,0.53}{\textbf{#1}}}
\newcommand{\ImportTok}[1]{#1}
\newcommand{\InformationTok}[1]{\textcolor[rgb]{0.56,0.35,0.01}{\textbf{\textit{#1}}}}
\newcommand{\KeywordTok}[1]{\textcolor[rgb]{0.13,0.29,0.53}{\textbf{#1}}}
\newcommand{\NormalTok}[1]{#1}
\newcommand{\OperatorTok}[1]{\textcolor[rgb]{0.81,0.36,0.00}{\textbf{#1}}}
\newcommand{\OtherTok}[1]{\textcolor[rgb]{0.56,0.35,0.01}{#1}}
\newcommand{\PreprocessorTok}[1]{\textcolor[rgb]{0.56,0.35,0.01}{\textit{#1}}}
\newcommand{\RegionMarkerTok}[1]{#1}
\newcommand{\SpecialCharTok}[1]{\textcolor[rgb]{0.81,0.36,0.00}{\textbf{#1}}}
\newcommand{\SpecialStringTok}[1]{\textcolor[rgb]{0.31,0.60,0.02}{#1}}
\newcommand{\StringTok}[1]{\textcolor[rgb]{0.31,0.60,0.02}{#1}}
\newcommand{\VariableTok}[1]{\textcolor[rgb]{0.00,0.00,0.00}{#1}}
\newcommand{\VerbatimStringTok}[1]{\textcolor[rgb]{0.31,0.60,0.02}{#1}}
\newcommand{\WarningTok}[1]{\textcolor[rgb]{0.56,0.35,0.01}{\textbf{\textit{#1}}}}
\usepackage{graphicx}
\makeatletter
\def\maxwidth{\ifdim\Gin@nat@width>\linewidth\linewidth\else\Gin@nat@width\fi}
\def\maxheight{\ifdim\Gin@nat@height>\textheight\textheight\else\Gin@nat@height\fi}
\makeatother
% Scale images if necessary, so that they will not overflow the page
% margins by default, and it is still possible to overwrite the defaults
% using explicit options in \includegraphics[width, height, ...]{}
\setkeys{Gin}{width=\maxwidth,height=\maxheight,keepaspectratio}
% Set default figure placement to htbp
\makeatletter
\def\fps@figure{htbp}
\makeatother
\setlength{\emergencystretch}{3em} % prevent overfull lines
\providecommand{\tightlist}{%
  \setlength{\itemsep}{0pt}\setlength{\parskip}{0pt}}
\setcounter{secnumdepth}{-\maxdimen} % remove section numbering
\ifLuaTeX
  \usepackage{selnolig}  % disable illegal ligatures
\fi
\usepackage{bookmark}
\IfFileExists{xurl.sty}{\usepackage{xurl}}{} % add URL line breaks if available
\urlstyle{same}
\hypersetup{
  pdftitle={Tinder Analysis},
  pdfauthor={Ethan Gandiboyina, Sean Paulo, Daniel Mon , Nhi Nguyen},
  hidelinks,
  pdfcreator={LaTeX via pandoc}}

\title{Tinder Analysis}
\author{Ethan Gandiboyina, Sean Paulo, Daniel Mon , Nhi Nguyen}
\date{2025-01-22}

\begin{document}
\maketitle

\begin{Shaded}
\begin{Highlighting}[]
\FunctionTok{library}\NormalTok{(FactoMineR)}
\FunctionTok{library}\NormalTok{(factoextra)}
\end{Highlighting}
\end{Shaded}

\begin{verbatim}
## Loading required package: ggplot2
\end{verbatim}

\begin{verbatim}
## Welcome! Want to learn more? See two factoextra-related books at https://goo.gl/ve3WBa
\end{verbatim}

\begin{Shaded}
\begin{Highlighting}[]
\FunctionTok{library}\NormalTok{(dplyr)}
\end{Highlighting}
\end{Shaded}

\begin{verbatim}
## 
## Attaching package: 'dplyr'
\end{verbatim}

\begin{verbatim}
## The following objects are masked from 'package:stats':
## 
##     filter, lag
\end{verbatim}

\begin{verbatim}
## The following objects are masked from 'package:base':
## 
##     intersect, setdiff, setequal, union
\end{verbatim}

\begin{Shaded}
\begin{Highlighting}[]
\FunctionTok{library}\NormalTok{(ggplot2)}
\FunctionTok{library}\NormalTok{(corrplot)}
\end{Highlighting}
\end{Shaded}

\begin{verbatim}
## corrplot 0.95 loaded
\end{verbatim}

\begin{Shaded}
\begin{Highlighting}[]
\FunctionTok{library}\NormalTok{(GGally)}
\end{Highlighting}
\end{Shaded}

\begin{verbatim}
## Registered S3 method overwritten by 'GGally':
##   method from   
##   +.gg   ggplot2
\end{verbatim}

\begin{Shaded}
\begin{Highlighting}[]
\FunctionTok{library}\NormalTok{(caret)}
\end{Highlighting}
\end{Shaded}

\begin{verbatim}
## Loading required package: lattice
\end{verbatim}

\begin{Shaded}
\begin{Highlighting}[]
\FunctionTok{library}\NormalTok{(ggsci)}
\FunctionTok{library}\NormalTok{(mclust)}
\end{Highlighting}
\end{Shaded}

\begin{verbatim}
## Package 'mclust' version 6.1.1
## Type 'citation("mclust")' for citing this R package in publications.
\end{verbatim}

\begin{Shaded}
\begin{Highlighting}[]
\FunctionTok{library}\NormalTok{(pROC)}
\end{Highlighting}
\end{Shaded}

\begin{verbatim}
## Type 'citation("pROC")' for a citation.
\end{verbatim}

\begin{verbatim}
## 
## Attaching package: 'pROC'
\end{verbatim}

\begin{verbatim}
## The following objects are masked from 'package:stats':
## 
##     cov, smooth, var
\end{verbatim}

\begin{Shaded}
\begin{Highlighting}[]
\FunctionTok{library}\NormalTok{(plotly)}
\end{Highlighting}
\end{Shaded}

\begin{verbatim}
## 
## Attaching package: 'plotly'
\end{verbatim}

\begin{verbatim}
## The following object is masked from 'package:ggplot2':
## 
##     last_plot
\end{verbatim}

\begin{verbatim}
## The following object is masked from 'package:stats':
## 
##     filter
\end{verbatim}

\begin{verbatim}
## The following object is masked from 'package:graphics':
## 
##     layout
\end{verbatim}

\begin{Shaded}
\begin{Highlighting}[]
\NormalTok{tinder}\OtherTok{\textless{}{-}}\FunctionTok{read.csv}\NormalTok{(}\StringTok{"D:/School Stuff/Tinder Analysis/Tinder\_Analysis/tinder\_users.db {-} pays.csv"}\NormalTok{)}

\FunctionTok{str}\NormalTok{(tinder)}
\end{Highlighting}
\end{Shaded}

\begin{verbatim}
## 'data.frame':    3000 obs. of  17 variables:
##  $ userid         : int  1 2 3 4 5 6 7 8 9 10 ...
##  $ date.crea      : chr  "9/17/2011" "1/17/2017" "5/14/2019" "11/27/2015" ...
##  $ score          : num  1.5 8.95 2.5 2.82 2.12 ...
##  $ n.matches      : int  11 56 13 32 21 14 10 9 6 31 ...
##  $ n.updates.photo: int  5 2 3 5 1 2 1 1 -1 2 ...
##  $ n.photos       : int  6 6 4 2 4 6 4 3 4 5 ...
##  $ last.connex    : chr  "10/7/2011" "1/31/2017" "6/17/2019" "1/15/2016" ...
##  $ last.up.photo  : chr  "10/2/2011" "2/3/2017" "6/19/2019" "12/9/2015" ...
##  $ last.pr.update : logi  NA NA NA NA NA NA ...
##  $ gender         : int  1 1 1 0 0 1 0 0 1 1 ...
##  $ sent.ana       : num  6.49 4.59 6.47 5.37 5.57 ...
##  $ length.prof    : num  0 20.7 31.4 0 38.5 ...
##  $ voyage         : int  0 0 0 0 0 0 1 1 0 1 ...
##  $ laugh          : int  0 0 0 0 1 0 0 0 0 0 ...
##  $ photo.elevator : int  0 0 0 0 0 0 0 0 0 0 ...
##  $ photo.beach    : int  0 1 1 1 0 0 1 0 0 0 ...
##  $ Country        : chr  "France" "Germany" "England" "France" ...
\end{verbatim}

Subsetting the data to preform MCA and PCA:

\begin{Shaded}
\begin{Highlighting}[]
\CommentTok{\# Removing date columns that won\textquotesingle{}t be used for analysis}
\NormalTok{date\_cols }\OtherTok{\textless{}{-}} \FunctionTok{c}\NormalTok{(}\StringTok{"last.up.photo"}\NormalTok{, }\StringTok{"last.pr.update"}\NormalTok{, }\StringTok{"last.connex"}\NormalTok{, }\StringTok{"date.crea"}\NormalTok{,}\StringTok{"Country"}\NormalTok{,}\StringTok{"userid"}\NormalTok{)}
\NormalTok{tinder }\OtherTok{\textless{}{-}}\NormalTok{ tinder }\SpecialCharTok{\%\textgreater{}\%} 
  \FunctionTok{select}\NormalTok{(}\SpecialCharTok{{-}}\FunctionTok{all\_of}\NormalTok{(date\_cols))}

\CommentTok{\# Define the qualitative variable names}
\NormalTok{qualitative }\OtherTok{\textless{}{-}} \FunctionTok{c}\NormalTok{( }\StringTok{"photo.beach"}\NormalTok{, }\StringTok{"photo.elevator"}\NormalTok{, }\StringTok{"laugh"}\NormalTok{, }\StringTok{"voyage"}\NormalTok{, }\StringTok{"gender"}\NormalTok{)}

\CommentTok{\# Select qualitative columns from the \textasciigrave{}tinder\textasciigrave{} DataFrame}
\NormalTok{tinder\_qualitative }\OtherTok{\textless{}{-}}\NormalTok{ tinder }\SpecialCharTok{\%\textgreater{}\%} 
  \FunctionTok{select}\NormalTok{(}\FunctionTok{all\_of}\NormalTok{(qualitative))}

\CommentTok{\# Select quantitative columns from the \textasciigrave{}tinder\textasciigrave{} DataFrame}
\NormalTok{tinder\_quantitative }\OtherTok{\textless{}{-}}\NormalTok{ tinder }\SpecialCharTok{\%\textgreater{}\%}
  \FunctionTok{select}\NormalTok{(}\SpecialCharTok{{-}}\FunctionTok{all\_of}\NormalTok{(qualitative))}


\FunctionTok{head}\NormalTok{(tinder\_quantitative)}
\end{Highlighting}
\end{Shaded}

\begin{verbatim}
##      score n.matches n.updates.photo n.photos sent.ana length.prof
## 1 1.495834        11               5        6 6.490446     0.00000
## 2 8.946863        56               2        6 4.589125    20.72286
## 3 2.496199        13               3        4 6.473182    31.39928
## 4 2.823579        32               5        2 5.368982     0.00000
## 5 2.117433        21               1        4 5.573949    38.51022
## 6 1.700014        14               2        6 5.464667    23.11221
\end{verbatim}

\begin{Shaded}
\begin{Highlighting}[]
\NormalTok{tinder\_qualitative\_df }\OtherTok{\textless{}{-}} \FunctionTok{as.data.frame}\NormalTok{(tinder\_qualitative)}
\NormalTok{tinder\_quantitative\_df }\OtherTok{\textless{}{-}} \FunctionTok{as.data.frame}\NormalTok{(tinder\_quantitative)}
\end{Highlighting}
\end{Shaded}

\begin{Shaded}
\begin{Highlighting}[]
\CommentTok{\# Convert columns to factors if they aren\textquotesingle{}t already}
\NormalTok{tinder\_qualitative }\OtherTok{\textless{}{-}}\NormalTok{ tinder\_qualitative }\SpecialCharTok{\%\textgreater{}\%}
  \FunctionTok{mutate}\NormalTok{(}\FunctionTok{across}\NormalTok{(}\FunctionTok{everything}\NormalTok{(), as.factor))}
\end{Highlighting}
\end{Shaded}

Preforming a pca on the quantitative variables of the data set:

\begin{Shaded}
\begin{Highlighting}[]
\NormalTok{tinder\_pca}\OtherTok{\textless{}{-}}\FunctionTok{prcomp}\NormalTok{(tinder\_quantitative,}\AttributeTok{scale=}\ConstantTok{TRUE}\NormalTok{)}
\end{Highlighting}
\end{Shaded}

Seeing the number of principle components needed to do analysis:

\begin{Shaded}
\begin{Highlighting}[]
\CommentTok{\# Compute variance and proportion of variance explained}
\NormalTok{tinder\_var }\OtherTok{\textless{}{-}}\NormalTok{ tinder\_pca}\SpecialCharTok{$}\NormalTok{sdev}\SpecialCharTok{\^{}}\DecValTok{2}
\NormalTok{tinder\_pve }\OtherTok{\textless{}{-}}\NormalTok{ tinder\_var }\SpecialCharTok{/} \FunctionTok{sum}\NormalTok{(tinder\_var)}

\CommentTok{\# Create elbow plot}
\FunctionTok{plot}\NormalTok{(tinder\_pve, }
     \AttributeTok{type =} \StringTok{"b"}\NormalTok{, }
     \AttributeTok{xlab =} \StringTok{"\# of components"}\NormalTok{, }
     \AttributeTok{ylab =} \StringTok{"\% Variance explained"}\NormalTok{, }
     \AttributeTok{ylim =} \FunctionTok{c}\NormalTok{(}\DecValTok{0}\NormalTok{, }\DecValTok{1}\NormalTok{), }
     \AttributeTok{main =} \StringTok{"Elbow Plot of PCA"}\NormalTok{, }
     \AttributeTok{pch =} \DecValTok{19}\NormalTok{, }\AttributeTok{col =} \StringTok{"blue"}\NormalTok{)}
\end{Highlighting}
\end{Shaded}

\includegraphics{tinder_analysis_files/figure-latex/unnamed-chunk-5-1.pdf}
The elbow plot doesn't show a clear inflection point. However, the
variance explained by 2 principle components only sums to about 60\%.
This suggests that we shouldn't reduce the dimensions of the dataset.

Preforming MCA:

\begin{Shaded}
\begin{Highlighting}[]
\CommentTok{\# Perform MCA}
\FunctionTok{head}\NormalTok{(tinder\_qualitative)}
\end{Highlighting}
\end{Shaded}

\begin{verbatim}
##   photo.beach photo.elevator laugh voyage gender
## 1           0              0     0      0      1
## 2           1              0     0      0      1
## 3           1              0     0      0      1
## 4           1              0     0      0      0
## 5           0              0     1      0      0
## 6           0              0     0      0      1
\end{verbatim}

\begin{Shaded}
\begin{Highlighting}[]
\NormalTok{tinder\_mca }\OtherTok{\textless{}{-}} \FunctionTok{MCA}\NormalTok{(tinder\_qualitative, }\AttributeTok{graph =} \ConstantTok{FALSE}\NormalTok{)}
\CommentTok{\#projecting components on pc 1 and 2}
\FunctionTok{fviz\_pca\_var}\NormalTok{(tinder\_mca, }\AttributeTok{col.var =} \StringTok{"red"}\NormalTok{,}\AttributeTok{xlab=}\StringTok{\textquotesingle{}PC 1\textquotesingle{}}\NormalTok{,}\AttributeTok{ylab=}\StringTok{\textquotesingle{}PC 2\textquotesingle{}}\NormalTok{)}
\end{Highlighting}
\end{Shaded}

\includegraphics{tinder_analysis_files/figure-latex/unnamed-chunk-6-1.pdf}

Here we several vectors whose magnitude projection onto an axis
represents the contribution a factor has on a specific principle
component. Here we see values such Gender\_1 and photo.beach\_1 having
significance on PC 1 while having no bearing on PC 2. Vectors with
smaller magnitudes aren't significant in the graphed PC's, but this
doesn't mean that they're irrelevant\ldots{}

To illustrate this we can print the contribution table:

\begin{Shaded}
\begin{Highlighting}[]
\NormalTok{variable\_contributions}\OtherTok{\textless{}{-}}\NormalTok{tinder\_mca}\SpecialCharTok{$}\NormalTok{var}\SpecialCharTok{$}\NormalTok{contrib}

\CommentTok{\# Subsetting the first 2 pc\textquotesingle{}s and converting to a df to print neatly}
\NormalTok{contribution\_table }\OtherTok{\textless{}{-}} \FunctionTok{as.data.frame}\NormalTok{(variable\_contributions[, }\DecValTok{1}\SpecialCharTok{:}\DecValTok{5}\NormalTok{])}
\FunctionTok{print}\NormalTok{(contribution\_table, }\AttributeTok{row.names =} \ConstantTok{TRUE}\NormalTok{)}
\end{Highlighting}
\end{Shaded}

\begin{verbatim}
##                       Dim 1        Dim 2      Dim 3       Dim 4       Dim 5
## photo.beach_0     5.4946973  0.003399576  0.2758844  0.03539420  5.95022575
## photo.beach_1    26.5132483  0.016403777  1.3312091  0.17078561 28.71128346
## photo.elevator_0  3.5745436  0.097037421  0.3207350  0.59363478  9.51946387
## photo.elevator_1 18.7199073  0.508185580  1.6796910  3.10886904 49.85349168
## laugh_0           0.1354581  5.810312265  6.7772077  6.74330511  0.09904889
## laugh_1           0.5568318 23.884639684 27.8592884 27.71992373  0.40716348
## voyage_0          0.3289347  8.641332018  0.5184313 11.93898139  0.57345266
## voyage_1          1.1549812 30.342120694  1.8203566 41.92108505  2.01355183
## gender_0         21.6338805  0.003883415  0.1540573  0.03549821  0.20269683
## gender_1         20.7507929  1.062594971  3.6202096  0.13053197  0.01801013
## gender_2          1.1367242 29.630090600 55.6429296  7.60199090  2.65161141
\end{verbatim}

\begin{Shaded}
\begin{Highlighting}[]
\FunctionTok{fviz\_contrib}\NormalTok{(tinder\_mca, }\AttributeTok{choice =} \StringTok{"var"}\NormalTok{, }\AttributeTok{axes =} \DecValTok{1}\SpecialCharTok{:}\DecValTok{5}\NormalTok{)}
\end{Highlighting}
\end{Shaded}

\includegraphics{tinder_analysis_files/figure-latex/unnamed-chunk-7-1.pdf}
This table shows the percentage of information contributed to each PC by
each variable. Going off of the variable Gender\_2, we can see that it
composes \textasciitilde1\% of PC1 but 55\% of pc2's variance.

Looking more broadly, we can see the \% of variance explained by each
principle component. This essentially tells us how many pc's are needed
to model the original data. In this case, if we used 5 pc's we could
model the original data with up to 86\% accuracy.

\begin{Shaded}
\begin{Highlighting}[]
\NormalTok{inertia}\OtherTok{\textless{}{-}}\FunctionTok{as.data.frame}\NormalTok{(tinder\_mca}\SpecialCharTok{$}\NormalTok{eig)}
\FunctionTok{print}\NormalTok{(inertia)}
\end{Highlighting}
\end{Shaded}

\begin{verbatim}
##       eigenvalue percentage of variance cumulative percentage of variance
## dim 1  0.2556924               21.30770                          21.30770
## dim 2  0.2040141               17.00117                          38.30887
## dim 3  0.1988266               16.56888                          54.87775
## dim 4  0.1967857               16.39880                          71.27655
## dim 5  0.1880749               15.67291                          86.94946
## dim 6  0.1566064               13.05054                         100.00000
\end{verbatim}

The inertia of the eigenvalues/principle components are relatively
similar to the pca.

\begin{Shaded}
\begin{Highlighting}[]
\CommentTok{\# Assuming \textquotesingle{}tinder\_mca\textquotesingle{} contains your MCA analysis result}
\FunctionTok{fviz\_screeplot}\NormalTok{(tinder\_mca, }\AttributeTok{addlabels =} \ConstantTok{TRUE}\NormalTok{, }\AttributeTok{main =} \StringTok{"Elbow Plot for MCA"}\NormalTok{,}\AttributeTok{ylim=}\FunctionTok{c}\NormalTok{(}\DecValTok{0}\NormalTok{,}\DecValTok{30}\NormalTok{))}
\end{Highlighting}
\end{Shaded}

\includegraphics{tinder_analysis_files/figure-latex/unnamed-chunk-9-1.pdf}

\begin{Shaded}
\begin{Highlighting}[]
\FunctionTok{head}\NormalTok{((tinder\_mca}\SpecialCharTok{$}\NormalTok{ind}\SpecialCharTok{$}\NormalTok{coord))}
\end{Highlighting}
\end{Shaded}

\begin{verbatim}
##        Dim 1      Dim 2       Dim 3     Dim 4      Dim 5
## 1  0.2610285 -0.3482267  0.09668760 0.0506516 -0.2076804
## 2  0.9319911 -0.3649160 -0.05365775 0.1045024  0.4905410
## 3  0.9319911 -0.3649160 -0.05365775 0.1045024  0.4905410
## 4  0.3439451 -0.3021685  0.09417473 0.1159881  0.4534615
## 5 -0.4208131  0.3288199 -0.41892582 0.7239216 -0.3249656
## 6  0.2610285 -0.3482267  0.09668760 0.0506516 -0.2076804
\end{verbatim}

A factor mapping on the PC's show us the significance of each factor on
a specific principle component. In real life this tells us that having a
laughing photo in your profile doesn't significantly impact the score of
the profile, whereas having a photo from a trip, or beach are more
likely to impact profile score.

Factor mapping of qualitative variables: Plotting the points of
individuals on a factorial plane:

\begin{Shaded}
\begin{Highlighting}[]
\CommentTok{\# Plot individuals}
\FunctionTok{fviz\_mca\_ind}\NormalTok{(tinder\_mca, }
             \AttributeTok{repel =} \ConstantTok{TRUE}\NormalTok{,        }\CommentTok{\# Avoid text overlap}
             \AttributeTok{col.ind =} \StringTok{"blue"}\NormalTok{,    }\CommentTok{\# Color of individuals}
             \AttributeTok{geom =} \StringTok{"point"}\NormalTok{,      }\CommentTok{\# Show points for individuals}
             \AttributeTok{title =} \StringTok{"Individuals on the Factorial Plane"}\NormalTok{)}
\end{Highlighting}
\end{Shaded}

\includegraphics{tinder_analysis_files/figure-latex/unnamed-chunk-10-1.pdf}
Now using our principle components we will cluster the data. In this
scenario, clustering will form groups consisting of profiles with
similar characteristics.

Looking at the graph above, the data seems to be grouped into four
groups. Due to the nature of this data, it seems that
k-means,hierarchical clustering would be the best method to cluster this
data.

Preforming K-means:

\begin{Shaded}
\begin{Highlighting}[]
\CommentTok{\#calculating the cluster using kmeans}
\FunctionTok{set.seed}\NormalTok{(}\DecValTok{321}\NormalTok{)}
\NormalTok{kmeans\_qual }\OtherTok{\textless{}{-}} \FunctionTok{kmeans}\NormalTok{(tinder\_mca}\SpecialCharTok{$}\NormalTok{ind}\SpecialCharTok{$}\NormalTok{coord, }\AttributeTok{centers =} \DecValTok{4}\NormalTok{)}

\CommentTok{\# Convert cluster assignments to factor (discrete variable)}
\NormalTok{kmeans\_clusters }\OtherTok{\textless{}{-}} \FunctionTok{factor}\NormalTok{(kmeans\_qual}\SpecialCharTok{$}\NormalTok{cluster)}

\CommentTok{\# Visualize K{-}means clusters on the MCA factorial plane}
\FunctionTok{fviz\_mca\_ind}\NormalTok{(tinder\_mca, }
             \AttributeTok{col.ind =}\NormalTok{ kmeans\_clusters,    }\CommentTok{\# Color points by clusters (discrete factor)}
             \AttributeTok{palette =} \StringTok{"jco"}\NormalTok{,              }\CommentTok{\# Choose color palette}
             \AttributeTok{addEllipses =} \ConstantTok{TRUE}\NormalTok{,           }\CommentTok{\# Add ellipses for clusters}
             \AttributeTok{repel =} \ConstantTok{TRUE}\NormalTok{,                 }\CommentTok{\# Avoid label overlap}
             \AttributeTok{label =} \StringTok{"none"}\NormalTok{,               }\CommentTok{\# Remove data labels}
             \AttributeTok{arrow =} \ConstantTok{FALSE}\NormalTok{,                }\CommentTok{\# Remove arrows}
             \AttributeTok{title =} \StringTok{"K{-}means Clustering on MCA"}\NormalTok{)}
\end{Highlighting}
\end{Shaded}

\includegraphics{tinder_analysis_files/figure-latex/unnamed-chunk-11-1.pdf}
This grouping doesn't look right, lets try hierarchical
clustering\ldots{}

\begin{Shaded}
\begin{Highlighting}[]
\CommentTok{\# Perform hierarchical clustering (using Euclidean distance and complete linkage)}
\NormalTok{dist\_mca }\OtherTok{\textless{}{-}} \FunctionTok{dist}\NormalTok{(tinder\_mca}\SpecialCharTok{$}\NormalTok{ind}\SpecialCharTok{$}\NormalTok{coord)  }\CommentTok{\# Compute distance matrix}
\NormalTok{hclust\_qual }\OtherTok{\textless{}{-}} \FunctionTok{hclust}\NormalTok{(dist\_mca, }\AttributeTok{method =} \StringTok{"complete"}\NormalTok{)  }\CommentTok{\# Perform hierarchical clustering}

\CommentTok{\# Cut the dendrogram to create 4 clusters}
\NormalTok{hclust\_clusters }\OtherTok{\textless{}{-}} \FunctionTok{cutree}\NormalTok{(hclust\_qual, }\AttributeTok{k =} \DecValTok{4}\NormalTok{)  }\CommentTok{\# }

\CommentTok{\# Convert hierarchical cluster assignments to factor (discrete variable)}
\NormalTok{hclust\_clusters\_factor }\OtherTok{\textless{}{-}} \FunctionTok{factor}\NormalTok{(hclust\_clusters)}

\CommentTok{\# Visualize hierarchical clustering on the MCA factorial plane}
\FunctionTok{fviz\_mca\_ind}\NormalTok{(tinder\_mca, }
             \AttributeTok{col.ind =}\NormalTok{ hclust\_clusters\_factor,    }\CommentTok{\# Color points by clusters}
             \AttributeTok{palette =} \StringTok{"jco"}\NormalTok{,                     }\CommentTok{\# Choose color palette}
             \AttributeTok{addEllipses =} \ConstantTok{TRUE}\NormalTok{,                  }\CommentTok{\# Add ellipses for clusters}
             \AttributeTok{repel =} \ConstantTok{TRUE}\NormalTok{,                        }\CommentTok{\# Avoid label overlap}
             \AttributeTok{label =} \StringTok{"none"}\NormalTok{,                      }\CommentTok{\# Remove data labels}
             \AttributeTok{arrow =} \ConstantTok{FALSE}\NormalTok{,                       }\CommentTok{\# Remove arrows}
             \AttributeTok{title =} \StringTok{"Hierarchical Clustering on MCA"}\NormalTok{)}
\end{Highlighting}
\end{Shaded}

\includegraphics{tinder_analysis_files/figure-latex/unnamed-chunk-12-1.pdf}
This gives a similar result to the kmeans clustering. To get a better
understanding on why the clustering is the way that it is, we can try
plotting the clusters using 3 pc's.

\begin{Shaded}
\begin{Highlighting}[]
\CommentTok{\# Extract the first three dimensions from the MCA coordinates}
\NormalTok{mca\_3d\_coords }\OtherTok{\textless{}{-}}\NormalTok{ tinder\_mca}\SpecialCharTok{$}\NormalTok{ind}\SpecialCharTok{$}\NormalTok{coord[, }\DecValTok{1}\SpecialCharTok{:}\DecValTok{3}\NormalTok{]  }\CommentTok{\# Assuming MCA has at least 3 dimensions}

\CommentTok{\# Create a 3D scatter plot}
\NormalTok{plot }\OtherTok{\textless{}{-}} \FunctionTok{plot\_ly}\NormalTok{(}
  \AttributeTok{x =} \SpecialCharTok{\textasciitilde{}}\NormalTok{mca\_3d\_coords[, }\DecValTok{1}\NormalTok{], }
  \AttributeTok{y =} \SpecialCharTok{\textasciitilde{}}\NormalTok{mca\_3d\_coords[, }\DecValTok{2}\NormalTok{], }
  \AttributeTok{z =} \SpecialCharTok{\textasciitilde{}}\NormalTok{mca\_3d\_coords[, }\DecValTok{3}\NormalTok{], }
  \AttributeTok{type =} \StringTok{"scatter3d"}\NormalTok{, }
  \AttributeTok{mode =} \StringTok{"markers"}\NormalTok{,}
  \AttributeTok{color =} \SpecialCharTok{\textasciitilde{}}\NormalTok{kmeans\_clusters,  }\CommentTok{\# Color by clusters}
  \AttributeTok{colors =} \StringTok{"Set2"}           \CommentTok{\# Color palette}
\NormalTok{)}

\CommentTok{\# Add layout}
\NormalTok{plot }\OtherTok{\textless{}{-}}\NormalTok{ plot }\SpecialCharTok{\%\textgreater{}\%} \FunctionTok{layout}\NormalTok{(}
  \AttributeTok{title =} \StringTok{"3D K{-}means Clustering on MCA"}\NormalTok{,}
  \AttributeTok{scene =} \FunctionTok{list}\NormalTok{(}
    \AttributeTok{xaxis =} \FunctionTok{list}\NormalTok{(}\AttributeTok{title =} \StringTok{"Dim 1"}\NormalTok{),}
    \AttributeTok{yaxis =} \FunctionTok{list}\NormalTok{(}\AttributeTok{title =} \StringTok{"Dim 2"}\NormalTok{),}
    \AttributeTok{zaxis =} \FunctionTok{list}\NormalTok{(}\AttributeTok{title =} \StringTok{"Dim 3"}\NormalTok{)}
\NormalTok{  )}
\NormalTok{)}

\CommentTok{\# Print the plot}
\NormalTok{plot}
\end{Highlighting}
\end{Shaded}

\begin{verbatim}
## file:///C:\Users\ethan\AppData\Local\Temp\RtmpO2Bxx4\file46a04311406d\widget46a055704717.html screenshot completed
\end{verbatim}

\includegraphics{tinder_analysis_files/figure-latex/unnamed-chunk-13-1.pdf}
In this graph we see that in 3 dimensions that the data doesn't follow
the pattern that graphing in 2d followed. *Note this graph actually
contains 3000 observations. Since the data of the MCA is formed by
factors, there are many points that overlap.

Now lets analyse on the PCA:

Lets what quantitative variables are correlated:

\begin{Shaded}
\begin{Highlighting}[]
\NormalTok{ coor\_plot}\OtherTok{\textless{}{-}}\FunctionTok{ggcorr}\NormalTok{(}
   \AttributeTok{data=}\NormalTok{tinder\_quantitative,}
   \AttributeTok{label=}\ConstantTok{TRUE}\NormalTok{)}
 \FunctionTok{print}\NormalTok{(coor\_plot)}
\end{Highlighting}
\end{Shaded}

\includegraphics{tinder_analysis_files/figure-latex/unnamed-chunk-14-1.pdf}
The only variables that have a significant relationship are the
``n.matches'' and the ``score''. This is since score is in part
calculated by using n matches.

Using the MCA we can visualize the contribution of each qualitative
variable on the pc's:

\begin{Shaded}
\begin{Highlighting}[]
\FunctionTok{fviz\_pca\_var}\NormalTok{(}
\NormalTok{  tinder\_pca,}\AttributeTok{col.var =} \StringTok{"contrib"}\NormalTok{,}\AttributeTok{gradient.cols=}\FunctionTok{c}\NormalTok{(}\StringTok{"blue"}\NormalTok{,}\StringTok{"purple"}\NormalTok{,}\StringTok{"red"}\NormalTok{),}\AttributeTok{rapel=}\ConstantTok{TRUE}
\NormalTok{)}
\end{Highlighting}
\end{Shaded}

\includegraphics{tinder_analysis_files/figure-latex/unnamed-chunk-15-1.pdf}

Here we can once again see the percentge of varience explained by the
individual pc's

\begin{Shaded}
\begin{Highlighting}[]
\NormalTok{inertia\_pca}\OtherTok{\textless{}{-}}\FunctionTok{as.data.frame}\NormalTok{(tinder\_pca}\SpecialCharTok{$}\NormalTok{eig)}
\FunctionTok{print}\NormalTok{(inertia)}
\end{Highlighting}
\end{Shaded}

\begin{verbatim}
##       eigenvalue percentage of variance cumulative percentage of variance
## dim 1  0.2556924               21.30770                          21.30770
## dim 2  0.2040141               17.00117                          38.30887
## dim 3  0.1988266               16.56888                          54.87775
## dim 4  0.1967857               16.39880                          71.27655
## dim 5  0.1880749               15.67291                          86.94946
## dim 6  0.1566064               13.05054                         100.00000
\end{verbatim}

To group users, we can plot them on a plane of pc's. In doing this we
can hopefully see a pattern between user characteristics and the number
of matches they have.

Plotting People on a factorial plane:

\begin{Shaded}
\begin{Highlighting}[]
\CommentTok{\# Plot individuals on the first two dimensions (Dim 1 and Dim 2)}
\FunctionTok{fviz\_pca\_ind}\NormalTok{(}
\NormalTok{  tinder\_pca,}
  \AttributeTok{axes =} \FunctionTok{c}\NormalTok{(}\DecValTok{1}\NormalTok{, }\DecValTok{2}\NormalTok{),       }\CommentTok{\# Dimensions to plot}
  \AttributeTok{geom.ind =} \StringTok{"point"}\NormalTok{,   }\CommentTok{\# Use points for individuals}
  \AttributeTok{col.ind =}\NormalTok{ tinder}\SpecialCharTok{$}\NormalTok{n.matches,     }\CommentTok{\# Color by the quality of representation (cos2)}
  \AttributeTok{gradient.cols =} \FunctionTok{c}\NormalTok{(}\StringTok{"yellow"}\NormalTok{, }\StringTok{"orange"}\NormalTok{, }\StringTok{"red"}\NormalTok{),  }\CommentTok{\# Color gradient}
  \AttributeTok{repel =} \ConstantTok{TRUE}          \CommentTok{\# Avoid overlapping labels}
\NormalTok{) }\SpecialCharTok{+}
\FunctionTok{ggtitle}\NormalTok{(}\StringTok{"Individuals on the Factorial Plane (Dim 1 vs Dim 2)"}\NormalTok{)}
\end{Highlighting}
\end{Shaded}

\includegraphics{tinder_analysis_files/figure-latex/unnamed-chunk-17-1.pdf}
Here we can see that there is a positive correlation between dim 1 and
the user score which is represented by the color.

To get more information on users we can cluster them by all principle
components.

Lets see how many clusters we should use an elbow plot:

\begin{Shaded}
\begin{Highlighting}[]
\FunctionTok{str}\NormalTok{(tinder\_pca)}
\end{Highlighting}
\end{Shaded}

\begin{verbatim}
## List of 5
##  $ sdev    : num [1:6] 1.531 1.026 0.981 0.935 0.819 ...
##  $ rotation: num [1:6, 1:6] 0.5989 0.6116 0.3174 0.0032 0.4068 ...
##   ..- attr(*, "dimnames")=List of 2
##   .. ..$ : chr [1:6] "score" "n.matches" "n.updates.photo" "n.photos" ...
##   .. ..$ : chr [1:6] "PC1" "PC2" "PC3" "PC4" ...
##  $ center  : Named num [1:6] 1.95 16.78 2.07 3.52 5 ...
##   ..- attr(*, "names")= chr [1:6] "score" "n.matches" "n.updates.photo" "n.photos" ...
##  $ scale   : Named num [1:6] 1.08 10.1 1.53 1.71 2.24 ...
##   ..- attr(*, "names")= chr [1:6] "score" "n.matches" "n.updates.photo" "n.photos" ...
##  $ x       : num [1:3000, 1:6] 0.312 6.163 0.504 2.11 0.186 ...
##   ..- attr(*, "dimnames")=List of 2
##   .. ..$ : NULL
##   .. ..$ : chr [1:6] "PC1" "PC2" "PC3" "PC4" ...
##  - attr(*, "class")= chr "prcomp"
\end{verbatim}

\begin{Shaded}
\begin{Highlighting}[]
\NormalTok{pca\_coords }\OtherTok{\textless{}{-}} \FunctionTok{as.data.frame}\NormalTok{(tinder\_pca}\SpecialCharTok{$}\NormalTok{x[, }\DecValTok{1}\SpecialCharTok{:}\DecValTok{6}\NormalTok{])}
\FunctionTok{fviz\_nbclust}\NormalTok{(pca\_coords, kmeans, }\AttributeTok{method =} \StringTok{"wss"}\NormalTok{)  }\CommentTok{\# Elbow method}
\end{Highlighting}
\end{Shaded}

\includegraphics{tinder_analysis_files/figure-latex/unnamed-chunk-18-1.pdf}

I decided to use 6 clusters for the Gaussian Mixture and K-means model
so that we get moderate variation between clusters, while maintaining a
decent square error. At 6 clusters there is on average a squared error
of \textasciitilde3 matches.

\begin{Shaded}
\begin{Highlighting}[]
\CommentTok{\# Load required libraries}
\FunctionTok{library}\NormalTok{(FactoMineR)}
\FunctionTok{library}\NormalTok{(factoextra)}

\CommentTok{\# Perform K{-}means clustering}
\FunctionTok{set.seed}\NormalTok{(}\DecValTok{42}\NormalTok{)  }\CommentTok{\# Set seed for reproducibility}
\NormalTok{kmeans\_result }\OtherTok{\textless{}{-}} \FunctionTok{kmeans}\NormalTok{(pca\_coords, }\AttributeTok{centers =} \DecValTok{6}\NormalTok{, }\AttributeTok{nstart =} \DecValTok{250}\NormalTok{)  }\CommentTok{\# Adjust \textquotesingle{}centers\textquotesingle{} as needed}
\end{Highlighting}
\end{Shaded}

\begin{verbatim}
## Warning: did not converge in 10 iterations
\end{verbatim}

\begin{Shaded}
\begin{Highlighting}[]
\CommentTok{\# View clustering results}
\FunctionTok{print}\NormalTok{(kmeans\_result)}
\end{Highlighting}
\end{Shaded}

\begin{verbatim}
## K-means clustering with 6 clusters of sizes 400, 613, 450, 599, 523, 415
## 
## Cluster means:
##          PC1        PC2         PC3         PC4         PC5         PC6
## 1  2.7450964 -0.2532806 -0.04801216  0.33038799  0.27276275 -0.01576899
## 2 -1.0223461 -0.3415785 -0.90078490  0.44907999  0.08422322 -0.01472560
## 3  0.4817928 -0.1383272 -0.84011420 -0.94089828 -0.28145885  0.01510575
## 4 -1.3370468  0.7940561  0.54462685 -0.07391694  0.14623527 -0.02822632
## 5  0.6174048  0.9186106  0.53412828  0.14272677 -0.08436620  0.02872047
## 6 -0.5064069 -1.4051224  0.82857065 -0.03471531 -0.18686471  0.02511702
## 
## Clustering vector:
##    [1] 3 1 5 3 5 6 2 4 4 1 5 4 2 3 1 3 6 3 2 2 6 4 2 3 4 1 2 3 2 4 2 4 5 3 4 6 4
##   [38] 4 3 3 2 3 6 4 6 1 3 4 2 4 6 6 6 2 1 1 1 1 2 4 5 4 2 3 2 2 3 3 2 3 2 6 6 3
##   [75] 3 4 3 1 4 3 6 4 2 6 5 5 6 6 1 4 4 5 4 3 4 4 6 6 3 5 2 3 2 3 6 6 4 2 3 5 2
##  [112] 5 6 6 5 4 6 4 6 3 6 3 3 2 5 1 2 3 5 1 5 3 2 4 5 6 3 5 1 5 2 6 5 4 2 1 1 1
##  [149] 5 6 3 2 6 1 2 4 5 1 2 1 5 2 5 6 1 2 1 1 2 2 2 5 3 1 3 2 3 2 5 1 3 4 1 3 4
##  [186] 4 6 2 2 6 4 2 1 2 6 2 6 6 3 6 4 5 2 5 6 2 3 1 2 1 4 6 1 4 4 5 4 2 3 6 6 6
##  [223] 1 3 1 1 3 3 2 1 4 4 2 6 2 4 4 1 4 6 4 5 3 2 2 3 4 1 1 3 4 3 2 6 6 2 3 4 1
##  [260] 6 3 1 3 5 3 1 3 5 2 6 6 2 1 4 4 2 5 4 4 6 6 5 2 6 1 3 4 1 2 5 4 2 5 6 5 2
##  [297] 3 3 6 2 5 1 1 5 5 5 1 2 1 6 2 2 2 2 2 3 2 5 4 3 4 6 4 3 1 5 4 2 2 3 4 1 5
##  [334] 4 2 2 4 1 4 3 5 1 1 4 2 4 6 1 5 5 2 3 4 2 5 2 4 5 2 1 2 2 1 5 5 2 3 5 4 4
##  [371] 5 2 3 2 4 3 2 5 5 5 5 6 5 5 4 1 4 5 6 5 6 4 1 5 4 1 5 2 4 1 6 2 6 4 6 6 2
##  [408] 5 1 6 4 2 2 4 5 6 2 4 4 5 2 3 5 4 1 5 5 4 3 5 1 4 4 5 5 2 3 1 6 2 5 3 6 3
##  [445] 2 1 6 3 5 6 6 1 1 3 3 1 5 6 1 4 6 5 5 2 6 5 2 2 4 1 2 4 5 6 4 2 4 4 2 1 5
##  [482] 4 3 4 5 3 4 2 3 4 6 3 5 3 6 1 5 2 5 2 2 3 2 6 3 3 3 2 3 6 6 6 5 3 2 2 2 4
##  [519] 5 4 2 5 6 2 4 4 4 1 4 5 3 5 4 2 3 1 1 2 6 2 4 2 3 5 6 4 3 2 3 6 5 3 3 2 4
##  [556] 6 6 5 4 2 4 5 4 5 2 2 1 2 6 5 3 4 1 4 4 4 1 1 4 2 3 6 4 2 2 4 1 2 2 1 1 2
##  [593] 2 5 1 2 4 3 2 4 3 2 3 4 5 4 4 6 4 3 1 2 2 3 2 2 5 1 1 6 5 4 6 3 3 3 2 4 1
##  [630] 2 6 5 3 4 2 1 2 5 3 6 6 4 1 4 3 4 3 4 6 3 4 5 6 4 4 2 4 6 2 5 6 1 6 5 6 6
##  [667] 4 5 4 5 3 2 3 2 1 4 1 5 4 1 1 2 4 4 5 4 2 1 1 3 2 4 3 4 3 3 4 6 2 1 4 3 2
##  [704] 4 6 4 4 2 6 2 2 4 6 4 4 5 6 2 1 3 2 6 3 4 5 5 5 3 5 2 1 5 1 6 4 4 1 1 1 4
##  [741] 5 2 5 4 4 6 2 2 4 4 5 4 4 5 3 2 4 5 2 1 3 6 3 2 3 3 5 2 5 2 3 4 5 6 5 3 1
##  [778] 3 6 4 1 3 5 1 2 5 1 6 5 2 2 6 5 1 4 5 6 4 5 2 4 2 5 4 2 6 1 2 2 6 5 3 1 4
##  [815] 6 1 1 6 1 4 5 5 4 4 4 2 4 3 2 6 5 1 6 1 2 5 1 5 2 5 2 4 5 1 1 5 3 4 2 2 2
##  [852] 5 1 4 6 5 1 4 4 3 2 4 2 5 1 5 4 1 4 2 2 2 4 1 6 6 4 2 2 2 2 4 2 5 2 4 6 5
##  [889] 3 6 1 1 6 2 2 6 4 3 6 3 6 2 1 4 3 1 5 1 2 1 5 2 5 4 4 1 3 4 2 4 3 4 4 5 4
##  [926] 6 5 2 3 2 5 2 6 5 4 5 1 5 5 3 4 5 1 5 4 4 3 1 3 5 3 1 6 3 5 5 2 4 1 2 2 6
##  [963] 4 4 3 6 5 3 5 4 1 5 4 4 4 2 4 1 6 2 5 5 1 2 3 4 4 5 5 2 2 2 5 5 6 5 4 5 2
## [1000] 2 1 5 6 5 3 6 6 4 4 2 2 3 4 6 6 6 5 1 1 3 5 4 2 5 2 2 1 6 4 6 6 4 4 4 2 1
## [1037] 6 6 4 4 3 5 5 5 2 3 2 1 1 6 4 3 5 6 4 1 1 5 5 4 6 4 5 3 2 1 2 6 3 6 2 2 2
## [1074] 6 1 2 2 2 3 2 4 3 5 1 5 2 1 4 5 4 2 6 5 2 5 2 5 3 3 5 4 5 1 1 5 6 6 6 2 1
## [1111] 1 2 2 2 1 2 6 1 5 1 6 2 4 6 4 5 3 3 6 2 1 2 5 4 6 2 5 3 6 4 4 4 3 1 2 2 3
## [1148] 1 5 4 3 5 4 1 4 5 4 3 3 6 5 6 6 5 6 4 4 4 3 4 2 4 5 5 5 1 1 2 6 3 4 6 2 1
## [1185] 4 6 6 2 5 6 5 4 2 4 2 3 1 6 3 2 4 5 1 4 1 5 5 2 3 2 2 1 3 6 5 4 3 2 2 1 6
## [1222] 2 4 2 1 5 3 6 2 4 2 2 4 4 6 4 4 3 2 4 5 5 4 2 4 1 5 2 2 4 3 3 6 2 6 2 5 3
## [1259] 4 5 2 6 5 4 5 5 3 5 3 1 6 4 3 5 6 1 5 4 3 1 6 6 4 5 4 6 5 5 6 3 4 5 2 1 6
## [1296] 6 5 1 4 4 3 5 3 4 2 4 3 2 4 2 5 4 5 3 5 3 2 3 6 6 4 4 2 2 5 5 2 5 3 1 4 4
## [1333] 2 4 1 3 3 2 6 4 2 1 4 5 2 5 5 4 2 1 4 5 6 4 5 4 2 2 4 4 1 2 5 6 4 1 1 5 4
## [1370] 6 6 1 3 5 3 6 2 2 3 1 4 3 3 5 6 2 4 4 2 3 4 4 6 5 2 2 3 5 6 3 2 1 2 6 6 2
## [1407] 2 2 6 5 2 5 2 4 4 5 6 4 2 2 5 5 5 6 6 3 1 4 5 5 5 1 4 5 2 5 1 6 6 1 6 6 3
## [1444] 2 6 3 3 5 4 4 2 5 5 4 5 2 3 5 3 2 1 5 3 6 3 5 4 2 4 5 4 5 2 2 2 1 5 6 2 2
## [1481] 1 3 2 6 1 2 6 3 6 4 5 4 2 2 3 6 3 2 3 1 1 2 2 1 3 2 2 2 5 3 5 1 2 3 5 2 5
## [1518] 2 4 5 6 3 1 2 2 3 1 5 6 1 1 3 3 4 3 3 3 6 2 3 2 3 4 5 4 5 2 1 2 1 4 1 4 6
## [1555] 4 3 2 4 6 5 6 4 3 4 2 5 2 4 4 2 2 5 3 3 5 4 1 3 4 4 5 2 1 2 2 4 6 1 6 1 6
## [1592] 1 2 2 4 4 4 1 2 2 4 5 4 4 1 5 6 6 5 4 5 4 4 3 4 4 1 4 2 1 2 5 5 1 3 6 4 4
## [1629] 5 3 1 5 5 2 5 2 2 6 4 3 5 5 6 4 3 1 3 1 3 4 4 4 3 2 4 1 3 5 5 4 3 6 4 6 3
## [1666] 3 4 4 1 3 3 2 3 2 3 3 3 2 1 5 1 4 5 1 5 1 2 4 6 1 1 2 4 2 5 2 5 2 6 1 3 2
## [1703] 1 4 3 4 2 3 4 1 3 4 5 6 6 4 3 6 4 5 2 1 5 6 3 5 5 6 1 2 6 1 6 4 6 2 4 4 1
## [1740] 1 5 6 2 4 5 1 5 1 1 6 4 1 1 4 5 4 5 6 4 5 5 4 5 6 4 4 5 4 1 1 5 2 4 4 6 5
## [1777] 2 4 6 4 1 2 2 1 1 3 3 2 4 4 5 2 4 1 1 1 3 2 5 4 6 2 4 3 5 4 4 5 1 5 5 5 5
## [1814] 2 2 5 2 4 3 5 3 2 4 5 4 2 2 4 4 3 4 5 6 2 3 4 4 2 2 4 3 5 4 3 4 2 2 2 5 1
## [1851] 6 3 5 3 6 3 6 2 5 2 3 2 3 2 3 5 5 6 2 1 6 4 3 3 4 5 4 1 4 6 5 2 6 5 6 2 4
## [1888] 2 3 1 1 3 1 4 1 2 6 4 1 6 3 2 3 2 6 2 4 4 2 6 4 4 4 4 5 2 2 5 3 1 5 6 3 5
## [1925] 2 5 6 6 5 2 6 3 6 1 2 4 1 4 2 4 3 6 2 2 3 5 5 2 2 5 5 4 1 5 6 3 2 2 2 5 3
## [1962] 6 3 2 2 3 2 6 4 6 5 3 6 6 5 2 5 1 1 2 4 1 6 2 6 4 1 2 3 5 3 6 4 3 4 3 2 2
## [1999] 3 4 3 3 5 2 4 4 2 4 4 2 6 1 6 1 5 6 2 1 5 3 2 4 5 3 2 1 3 4 3 6 5 5 4 3 6
## [2036] 2 1 1 5 6 6 3 3 5 3 2 4 6 3 6 1 1 1 3 6 5 3 4 3 4 5 6 3 2 3 6 2 5 3 3 4 4
## [2073] 5 4 5 3 2 6 4 4 6 1 6 3 5 5 2 2 4 2 3 4 6 3 3 2 1 4 4 1 2 1 4 1 4 2 6 6 3
## [2110] 4 5 1 1 6 3 2 5 3 3 2 6 2 1 5 1 5 4 2 6 2 2 2 1 4 1 3 5 6 2 5 2 2 2 6 4 2
## [2147] 4 6 5 1 6 4 5 3 2 2 1 2 5 3 3 5 1 5 4 2 2 6 6 2 5 4 5 6 2 4 4 2 3 6 2 6 2
## [2184] 4 2 6 4 5 5 4 6 1 3 3 2 4 4 6 4 6 2 2 2 2 4 6 1 4 1 6 5 4 3 5 3 6 5 3 4 5
## [2221] 6 4 4 2 3 4 4 4 5 2 4 6 6 6 6 4 4 2 3 6 3 3 2 2 3 4 6 3 1 6 4 2 4 1 6 3 4
## [2258] 4 3 1 6 5 6 4 3 3 1 6 3 1 6 5 5 3 1 1 3 2 4 3 3 1 2 2 1 6 1 6 2 3 3 1 1 2
## [2295] 4 3 4 5 2 3 5 2 6 1 4 2 3 5 4 2 2 2 3 4 5 4 1 1 5 1 5 6 4 5 6 6 5 5 4 2 2
## [2332] 1 2 6 2 6 4 3 2 4 3 6 4 2 4 6 5 3 5 4 4 4 3 5 2 1 3 5 6 6 4 5 1 2 3 3 6 2
## [2369] 2 1 3 1 3 3 5 3 4 5 5 1 5 5 4 5 1 2 3 6 3 1 3 4 4 2 2 1 5 5 2 3 2 6 1 2 5
## [2406] 5 1 4 3 1 4 2 1 5 3 2 1 2 2 5 4 4 4 1 5 2 5 3 2 4 4 2 1 4 6 3 5 4 2 3 2 1
## [2443] 2 6 6 5 5 6 3 2 4 4 4 2 2 4 4 1 5 1 4 6 1 3 2 3 2 2 5 5 1 2 5 1 6 4 2 1 4
## [2480] 6 3 4 6 5 5 2 4 6 1 1 4 3 5 2 2 5 5 5 2 1 6 2 3 6 4 5 1 3 4 2 3 5 4 4 2 6
## [2517] 5 4 3 3 4 4 1 3 4 5 1 3 3 2 2 1 6 6 6 6 5 2 2 4 5 3 5 4 2 6 2 6 4 1 4 3 3
## [2554] 3 3 3 3 3 2 6 4 2 5 2 4 6 2 1 4 2 5 1 4 1 4 6 5 5 4 6 2 1 2 4 5 6 3 3 1 4
## [2591] 3 5 3 3 2 3 2 4 6 2 3 5 4 4 3 4 4 4 4 6 2 5 5 4 2 5 3 2 2 1 1 2 5 5 6 2 2
## [2628] 4 2 4 3 4 4 6 5 4 2 6 4 2 1 2 6 5 4 4 4 5 2 2 3 6 1 1 2 2 1 2 3 2 2 3 5 5
## [2665] 5 1 5 3 5 1 5 2 3 4 6 4 2 2 4 5 2 2 5 2 1 2 2 1 5 4 6 4 6 2 3 4 6 3 6 3 2
## [2702] 4 1 4 1 6 6 1 2 4 3 6 3 1 2 2 3 5 5 1 3 1 1 5 5 5 5 1 3 1 5 5 6 1 1 3 6 3
## [2739] 2 2 3 3 1 4 5 5 5 1 2 1 3 5 6 1 2 2 1 3 1 5 5 1 3 1 5 3 4 5 1 1 4 4 2 4 3
## [2776] 1 6 3 4 1 5 6 2 6 3 2 4 4 6 3 2 4 5 1 6 6 1 3 5 2 6 2 2 3 1 2 1 6 5 4 2 1
## [2813] 5 2 6 2 6 3 2 4 4 2 6 4 1 4 5 2 3 4 4 4 3 2 5 1 5 4 2 3 6 5 6 5 6 4 3 3 2
## [2850] 5 6 5 1 4 3 3 5 5 4 2 3 1 5 5 1 5 4 4 2 5 6 2 5 6 4 4 1 5 3 2 2 4 2 5 4 6
## [2887] 5 4 5 2 3 1 4 6 4 2 5 4 4 2 2 5 2 6 4 3 6 6 6 5 3 3 2 3 6 6 5 3 3 2 2 1 6
## [2924] 5 5 6 6 2 4 4 6 5 4 4 1 3 1 3 5 2 4 2 5 4 5 4 3 2 5 2 5 4 5 2 1 5 4 6 1 6
## [2961] 4 4 1 2 4 3 4 2 2 1 5 2 4 1 5 4 3 5 6 5 2 1 5 5 2 4 3 4 2 2 5 3 1 3 2 3 5
## [2998] 1 1 5
## 
## Within cluster sum of squares by cluster:
## [1] 1877.886 1363.571 1453.163 1463.862 1423.827 1421.426
##  (between_SS / total_SS =  50.0 %)
## 
## Available components:
## 
## [1] "cluster"      "centers"      "totss"        "withinss"     "tot.withinss"
## [6] "betweenss"    "size"         "iter"         "ifault"
\end{verbatim}

\begin{Shaded}
\begin{Highlighting}[]
\CommentTok{\# Add clusters to the PCA result for visualization}
\FunctionTok{fviz\_pca\_ind}\NormalTok{(}
\NormalTok{  tinder\_pca,}
  \AttributeTok{geom.ind =} \StringTok{"point"}\NormalTok{,        }\CommentTok{\# Use points to represent individuals}
  \AttributeTok{col.ind =} \FunctionTok{as.factor}\NormalTok{(kmeans\_result}\SpecialCharTok{$}\NormalTok{cluster),  }\CommentTok{\# Color by clusters}
  \AttributeTok{palette =} \StringTok{"Set2"}\NormalTok{,          }\CommentTok{\# Use a color palette}
  \AttributeTok{addEllipses =} \ConstantTok{TRUE}\NormalTok{,        }\CommentTok{\# Add confidence ellipses for clusters}
  \AttributeTok{legend.title =} \StringTok{"Clusters"}  \CommentTok{\# Title for the legend}
\NormalTok{) }\SpecialCharTok{+}
\FunctionTok{ggtitle}\NormalTok{(}\StringTok{"K{-}Means Clustering on PCA Coordinates"}\NormalTok{)}
\end{Highlighting}
\end{Shaded}

\includegraphics{tinder_analysis_files/figure-latex/unnamed-chunk-19-1.pdf}

\begin{Shaded}
\begin{Highlighting}[]
\NormalTok{gmm\_result }\OtherTok{\textless{}{-}} \FunctionTok{Mclust}\NormalTok{(pca\_coords, }\AttributeTok{G =} \DecValTok{4}\NormalTok{)  }\CommentTok{\# assigning 4 clusters}
\NormalTok{pca\_coords}\SpecialCharTok{$}\NormalTok{Cluster }\OtherTok{\textless{}{-}} \FunctionTok{as.factor}\NormalTok{(gmm\_result}\SpecialCharTok{$}\NormalTok{classification)  }\CommentTok{\# For GMM}
\end{Highlighting}
\end{Shaded}

Comparing to another clustering method\ldots{}

\begin{Shaded}
\begin{Highlighting}[]
\FunctionTok{ggplot}\NormalTok{(pca\_coords, }\FunctionTok{aes}\NormalTok{(}\AttributeTok{x =}\NormalTok{ PC1, }\AttributeTok{y =}\NormalTok{ PC2, }\AttributeTok{color =}\NormalTok{ Cluster)) }\SpecialCharTok{+}
  \FunctionTok{geom\_point}\NormalTok{(}\AttributeTok{size =} \DecValTok{2}\NormalTok{, }\AttributeTok{alpha =} \FloatTok{0.6}\NormalTok{) }\SpecialCharTok{+}
  \FunctionTok{stat\_ellipse}\NormalTok{(}\FunctionTok{aes}\NormalTok{(}\AttributeTok{fill =}\NormalTok{ Cluster), }\AttributeTok{alpha =} \FloatTok{0.2}\NormalTok{, }\AttributeTok{geom =} \StringTok{"polygon"}\NormalTok{) }\SpecialCharTok{+}
  \FunctionTok{scale\_color\_brewer}\NormalTok{(}\AttributeTok{palette =} \StringTok{"Set1"}\NormalTok{) }\SpecialCharTok{+}  \CommentTok{\# Color scheme}
  \FunctionTok{scale\_fill\_brewer}\NormalTok{(}\AttributeTok{palette =} \StringTok{"Set1"}\NormalTok{) }\SpecialCharTok{+}   \CommentTok{\# Matching fill colors for ellipses}
  \FunctionTok{labs}\NormalTok{(}
    \AttributeTok{title =} \StringTok{"Gaussian Mixture Model Clusters with Ellipses"}\NormalTok{,}
    \AttributeTok{x =} \StringTok{"Principal Component 1 (PC1)"}\NormalTok{,}
    \AttributeTok{y =} \StringTok{"Principal Component 2 (PC2)"}\NormalTok{,}
    \AttributeTok{color =} \StringTok{"Cluster"}\NormalTok{,}
    \AttributeTok{fill =} \StringTok{"Cluster"}
\NormalTok{  ) }\SpecialCharTok{+}
  \FunctionTok{theme\_minimal}\NormalTok{() }\SpecialCharTok{+}
  \FunctionTok{theme}\NormalTok{(}
    \AttributeTok{plot.title =} \FunctionTok{element\_text}\NormalTok{(}\AttributeTok{hjust =} \FloatTok{0.5}\NormalTok{),}
    \AttributeTok{legend.position =} \StringTok{"right"}
\NormalTok{  )}
\end{Highlighting}
\end{Shaded}

\includegraphics{tinder_analysis_files/figure-latex/unnamed-chunk-21-1.pdf}
Now we can plot the average measurements for each cluster\ldots{}

\begin{Shaded}
\begin{Highlighting}[]
\NormalTok{cluster\_summary }\OtherTok{\textless{}{-}}\NormalTok{ tinder\_quantitative }\SpecialCharTok{\%\textgreater{}\%}
  \FunctionTok{group\_by}\NormalTok{(tinder\_quantitative}\SpecialCharTok{$}\NormalTok{Cluster) }\SpecialCharTok{\%\textgreater{}\%}
  \FunctionTok{summarise}\NormalTok{(}\FunctionTok{across}\NormalTok{(}\FunctionTok{where}\NormalTok{(is.numeric), mean, }\AttributeTok{na.rm =} \ConstantTok{TRUE}\NormalTok{))}\SpecialCharTok{\%\textgreater{}\%}
  \FunctionTok{arrange}\NormalTok{(}\FunctionTok{desc}\NormalTok{(n.matches))}
\end{Highlighting}
\end{Shaded}

\begin{verbatim}
## Warning: There was 1 warning in `summarise()`.
## i In argument: `across(where(is.numeric), mean, na.rm = TRUE)`.
## Caused by warning:
## ! The `...` argument of `across()` is deprecated as of dplyr 1.1.0.
## Supply arguments directly to `.fns` through an anonymous function instead.
## 
##   # Previously
##   across(a:b, mean, na.rm = TRUE)
## 
##   # Now
##   across(a:b, \(x) mean(x, na.rm = TRUE))
\end{verbatim}

\begin{Shaded}
\begin{Highlighting}[]
\FunctionTok{print}\NormalTok{(cluster\_summary)}
\end{Highlighting}
\end{Shaded}

\begin{verbatim}
## # A tibble: 1 x 6
##   score n.matches n.updates.photo n.photos sent.ana length.prof
##   <dbl>     <dbl>           <dbl>    <dbl>    <dbl>       <dbl>
## 1  1.95      16.8            2.07     3.52     5.00        16.1
\end{verbatim}

These clusters reinforce that the only quantitative factors (from this
dataset) that can be used to linearly predict the \# of matches of a
user is sent.ana, and n.updates.photo. However, it seems that there may
be a non linear relationship between n.photos and score.

\end{document}
